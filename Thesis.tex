%% ----------------------------------------------------------------
%% Thesis.tex -- MAIN FILE (the one that you compile with LaTeX)
%% ---------------------------------------------------------------- 

% Set up the document
\documentclass[a4paper, 11pt, oneside]{Thesis}  % Use the "Thesis" style, based on the ECS Thesis style by Steve Gunn

\usepackage{verbatim}
% Include any extra LaTeX packages required
\usepackage[square, numbers, comma, sort&compress]{natbib}  % Use the "Natbib" style for the references in the Bibliography

\usepackage[nottoc]{tocbibind} % bind bibliography to the table of contents
\usepackage{verbatim}  % Needed for the "comment" environment to make LaTeX comments
\usepackage{vector}  % Allows "\bvec{}" and "\buvec{}" for "blackboard" style bold vectors in maths
\usepackage[table]{xcolor}

\hypersetup{urlcolor=black, colorlinks=true}  % Colours hyperlinks in black, can be distracting if there are many links and colored blue.

%% ----------------------------------------------------------------

\begin{document}
\frontmatter      % Begin Roman style (i, ii, iii, iv...) page numbering

% Set up the Title Page
\title  {Kubernetes Infrastructure Management}
\authors  {Stephen O'Brien}
            
\addresses  {\groupname\\\deptname\\\univname}  % Do not change this here, instead these must be set in the "Thesis.cls" file, please look through it instead
\date       {\today}
\subject    {}
\keywords   {}

\maketitle
%% ----------------------------------------------------------------

\setstretch{1.3}  % It is better to have smaller font and larger line spacing than the other way round

% Define the page headers using the FancyHdr package and set up for one-sided printing
\fancyhead{}  % Clears all page headers and footers
\rhead{\thepage}  % Sets the right side header to show the page number
\lhead{}  % Clears the left side page header

\pagestyle{fancy}  % Finally, use the "fancy" page style to implement the FancyHdr headers
%% ----------------------------------------------------------------
% Declaration Page required for the Thesis, your institution may give you a different text to place here


\Declaration{

\addtocontents{toc}{\vspace{1em}}  % Add a gap in the Contents, for aesthetics

I, Stephen O'Brien, declare that this proposal titled, `Kubernetes Infrastructure Management' and the work presented in it are my own. I confirm that:

\begin{itemize} 
\item[\tiny{$\blacksquare$}] This work was done wholly or mainly while in candidature for an masters degree at Cork Institute of Technology.
 
\item[\tiny{$\blacksquare$}] Where I have consulted the published work of others, this is always clearly attributed.
 
\item[\tiny{$\blacksquare$}] Where I have quoted from the work of others, the source is always given. With the exception of such quotations, this project report is entirely my own work.
 
\item[\tiny{$\blacksquare$}] I have acknowledged all main sources of help.
 
\item[\tiny{$\blacksquare$}] I understand that my project documentation may be stored in the library at CIT, and may be referenced by others in the future.
\\
\end{itemize}
 
 
Signed:\\
\rule[1em]{25em}{0.5pt}  % This prints a line for the signature
 
Date:\\
\rule[1em]{25em}{0.5pt}  % This prints a line to write the date
}
\clearpage  % Declaration ended, now start a new page

%% ----------------------------------------------------------------
% The Abstract Page
\addtotoc{Abstract}  % Add the "Abstract" page entry to the Contents
\abstract{
\addtocontents{toc}{\vspace{1em}}  % Add a gap in the Contents, for aesthetics

Currently the initialization of private production-grade Kubernetes clusters requires the use of multiple tools to create and manage the underlying virtual resources. Public clouds such as Google Cloud or Amazon Web Services have tools which reduce this complexity and manage the lifecycle of nodes in the cluster. This paper outlines a Kubernetes-native infrastructure management system for private hypervisors, with the capability of managing the underlying virtual resources of a Kubernetes cluster on multiple hypervisors. Management capabilities include scaling, health monitoring and recovery, zero-downtime updates and querying of the underlying infrastructure.

}

\clearpage  % Abstract ended, start a new page
%% ----------------------------------------------------------------
\begin{comment}

\setstretch{1.3}  % Reset the line-spacing to 1.3 for body text (if it has changed)

% The Acknowledgements page, for thanking everyone
\acknowledgements{
\addtocontents{toc}{\vspace{1em}}  % Add a gap in the Contents, for aesthetics

The acknowledgements and the people to thank go here, don't forget to include your project adviser\ldots

}
\clearpage  % End of the Acknowledgements
%% ----------------------------------------------------------------

\pagestyle{fancy}  %The page style headers have been "empty" all this time, now use the "fancy" headers as defined before to bring them back


%% ----------------------------------------------------------------
\lhead{\emph{Contents}}  % Set the left side page header to "Contents"
\tableofcontents  % Write out the Table of Contents

%% ----------------------------------------------------------------
\lhead{\emph{List of Figures}}  % Set the left side page header to "List if Figures"
\listoffigures  % Write out the List of Figures

%% ----------------------------------------------------------------
\lhead{\emph{List of Tables}}  % Set the left side page header to "List of Tables"
\listoftables  % Write out the List of Tables

%% ----------------------------------------------------------------
\setstretch{1.5}  % Set the line spacing to 1.5, this makes the following tables easier to read
\clearpage  % Start a new page
\lhead{\emph{Abbreviations}}  % Set the left side page header to "Abbreviations"
\listofsymbols{ll}  % Include a list of Abbreviations (a table of two columns)
{
% \textbf{Acronym} & \textbf{W}hat (it) \textbf{S}tands \textbf{F}or \\
\textbf{LAH} & \textbf{L}ist \textbf{A}bbreviations \textbf{H}ere \\

}

%% ----------------------------------------------------------------
% End of the pre-able, contents and lists of things
% Begin the Dedication page

\setstretch{1.3}  % Return the line spacing back to 1.3

\pagestyle{empty}  % Page style needs to be empty for this page
\dedicatory{For/Dedicated to/To my\ldots}

\addtocontents{toc}{\vspace{2em}}  % Add a gap in the Contents, for aesthetics

%%
\end{comment} ----------------------------------------------------------------
\mainmatter	  % Begin normal, numeric (1,2,3...) page numbering
\pagestyle{fancy}  % Return the page headers back to the "fancy" style

\chapter{Research Context}
\lhead{\emph{Research Context}}
Describe the broad context of the research, including a review of the current state of the art in the topic of the proposed research with references, and the overall contribution which it will make to the general field of research.
 % Introduction

\chapter{Research Aim}
\lhead{\emph{Research Aim}}
Outline the goal or overarching purpose of the research project. % Background Theory 

\chapter{Research Objectives}
\lhead{\emph{Research Objectives}}

\begin{enumerate}
\item Identify a means of initializing a Kubernetes cluster which can recover in the event of node failure to any node.
\item Investigate how Kubernetes can apply updates to its underlying nodes with zero downtime to the workloads it is running or to its own components.
\item Investigate the addition of a custom API to Kubernetes for the creation and querying of custom resources necessary to define the state for objectives 1 and 2.
\end{enumerate} % Problem

%\chapter{Research Methodology}
\lhead{\emph{Research Methodology}}
Outline the goal or overarching purpose of the research project.
Describe the methodology to be used in the proposed research and why it is appropriate to the research objectives. % Solution Approach

%\chapter{Work Plan}
\lhead{\emph{Work Plan}}
Present the research work plan, outlining the main research tasks and timing, including a Gantt chart or equivalent.  % Conclusions and Term 2 work
%\chapter{Ethical Issues}
\lhead{\emph{Ethical Issues}}
If the proposed research directly involves human or live animal subjects, discuss the ethical issues involved and the actions that will be taken to ensure compliance with CIT ethics guidelines and with the CIT Child Protection Policy (if children are involved).
%% ----------------------------------------------------------------
\label{Bibliography}
\bibliographystyle{IEEEtranN}  % Use the "IEEE Transaction" BibTeX style for formatting the Bibliography
\bibliography{Bibliography}  % The references (bibliography) information are stored in the file named "Bibliography.bib"
\lhead{\emph{Bibliography}}  % Change the left side page header to "Bibliography"

%% ----------------------------------------------------------------
% Now begin the Appendices, including them as separate files

\addtocontents{toc}{\vspace{2em}} % Add a gap in the Contents, for aesthetics

\appendix % Cue to tell LaTeX that the following 'chapters' are Appendices

%\chapter{Code Snippets}

Put appendix material in this section e.g. code snippets	% Appendix Title

%\chapter{Wireframe Models} % Appendix Title

%\input{Appendices/AppendixC} % Appendix Title

\addtocontents{toc}{\vspace{2em}}  % Add a gap in the Contents, for aesthetics
\backmatter
\end{document}  % The End
%% ----------------------------------------------------------------